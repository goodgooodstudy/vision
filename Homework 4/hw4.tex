\documentclass[twoside]{article}

\usepackage{amsmath,amsthm,amssymb,graphicx}
\usepackage{hyperref}
\usepackage[numbers]{natbib}
\usepackage{float}
\usepackage{bbm}

\theoremstyle{definition}
\newtheorem{thm}{Theorem}[section]
\newtheorem{lem}[thm]{Lemma}
\newtheorem{prop}[thm]{Proposition}
\newtheorem{cor}[thm]{Corollary}
\newenvironment{pf}{{\noindent\sc Proof. }}{\qed}
\newenvironment{map}{\[\begin{array}{cccc}} {\end{array}\]}

\newcommand{\comment}[1]{}
\theoremstyle{definition}
\newtheorem*{defn}{Definition}
\newtheorem*{exmp}{Example}
\newtheorem*{prob}{Problem}

\theoremstyle{remark}
\newtheorem*{rem}{Remark}
\newtheorem*{note}{Note}
\newtheorem*{exer}{Exercise}

\setlength{\oddsidemargin}{0.25 in}
\setlength{\evensidemargin}{-0.25 in}
\setlength{\topmargin}{-0.6 in}
\setlength{\textwidth}{6.5 in}
\setlength{\textheight}{8.5 in}
\setlength{\headsep}{0.75 in}
\setlength{\parindent}{0 in}
\setlength{\parskip}{0.1 in}

\newcommand{\widgraph}[2]{\includegraphics[keepaspectratio,width=#1]{#2}}
\newcommand{\widgraphr}[3]{\includegraphics[keepaspectratio,width=#1,angle=#3]{#2}}

\newcommand{\lecture}[4]{
   \pagestyle{myheadings}
   \thispagestyle{plain}
   \newpage
   \setcounter{page}{1}
   \noindent
   \begin{center}
   \framebox{
      \vbox{\vspace{2mm}
    \hbox to 6.28in { {\bf Stat365/665 (Spring 2015) Data Mining and Machine Learning \hfill Lecture: #4} }
       \vspace{6mm}
       \hbox to 6.28in { {\Large \hfill #1  \hfill} }
       \vspace{6mm}
       \hbox to 6.28in { {\it Lecturer: #2 \hfill Scribe: #3} }
      \vspace{2mm}}
   }
   \end{center}
   \markboth{#1}{#1}
   \vspace*{4mm}
}


%%%%%%%
% Some commonly used notation
%%%%%%%

\def\R{{\mathbb R}}
\def\X{{\mathcal X}}
\def\Y{{\mathcal Y}}
\def\H{{\mathcal H}}
\def\E{{\mathbb E}}
\def\sign{{\rm sign}}

\newcommand{\percent}{$\%$}

\begin{document}
\textbf{Computer Vision, Homework 4}\\
\textbf{Leon Lixing Yu}\\


\section{Parametrizing Curves}

\subsection{Find $\alpha(s)$}
\[\alpha (s) = (\frac{1}{k} cos(s), \frac{1}{k} sin(s))\]
\[ s \in [-\pi, \pi]\]
\[\alpha'(s) = (-\frac{1}{k} sin(s), \frac{1}{k} cos(s))\]
\[\| \alpha'(s)\| = \sqrt{sin^2(s)+cos^2(s)} = \frac{1}{k}\]

\subsection{Find Frenet approximation to the circle}
\[\alpha''(s) = (-\frac{1}{k} cos(s), -\frac{1}{k} sin(s))\]
\[\alpha (s_0 + s) = \alpha (s_0) + s ( \alpha' (s_0)) + \frac{s^2}{2} \alpha''(s_0)\]
From here we can sub in $\alpha$, $\alpha'$, and $\alpha''$.\\

\subsection{Plot magnitude error}
Before plotting the magnitude, I wants to plot the scatter graph to get a better scene of what's going on, see figure below. 

\begin{figure}[H]
\centering
\includegraphics[width=120mm]{scatter.jpg}
\caption{The blue circle is our scatter of $\alpha(s)$, and the red parabola is our scatter of frenet approximation. $s \in [-\pi, \pi]$ In the picture shown, the parabola lies in the plane through $\alpha(s_0)$ orthogonal to $\alpha(s_0)$}
\end{figure}

\begin{figure}[H]
\centering
\includegraphics[width=120mm]{01.jpg}
\caption{The x-axis is the number of data points, and y-axis is the magnitude of error, we see that the peak error is around 40 when curvature is 0.1, meaning radius is 10}
\end{figure}

\begin{figure}[H]
\centering
\includegraphics[width=120mm]{1.jpg}
\caption{The x-axis is the number of data points, and y-axis is the magnitude of error, we see that the peak error is around 4 when curvature is 1, meaning radius is 1}
\end{figure}

\begin{figure}[H]
\centering
\includegraphics[width=120mm]{10.jpg}
\caption{The x-axis is the number of data points, and y-axis is the magnitude of error, we see that the peak error is around 0.4 when curvature is 10, meaning radius is 0.1}
\end{figure}

The results show that the error is inverse-proportional to the curvature. However, I think the magnitude does not change the fact that the frenet approximation is the best quadratic approximation to $\alpha$ at $s_0$.

\section{Geometric Insights into Shape from Texture}
\subsection{calculate $F_{*p}$}
We have:
\[F_{*p} (\pmb{v}) = \frac{d}{du} (F(\alpha(u))), u=0\]

Given:
\[F(\pmb{p}) = r(\pmb{p}) \pmb{p}\]
\[\Rightarrow \frac{d}{du} (F(\alpha(u))) = \frac{d}{du} (r(\alpha(u)) \alpha(u))
\]
\[=r'((\alpha(u))\alpha(u)+r(\alpha(u))\alpha'(u)\]
We also know that:
\[\alpha(u) = \pmb{p}, \alpha'(u) = \pmb{v}, u = 0\]
Therefore, we have:
\[\Rightarrow F_{*p} (\pmb{v}) = \nabla r \;\pmb{ p} + r \;\pmb{v}\]

\subsection{Prove argmax}
If $\pmb{v} = \nabla r / \| \nabla r \|$, we know that $\pmb{v} $ is tangential to $r$, so the angle between $r$ and $\pmb{v} $ is 0. Plug it back to the equation above:
\[\| F_{*p} (\pmb{v}) \| =\| \nabla r \;\pmb{ p} + r ( \nabla r / \| \nabla r \|) cos(0) \|\]
This form will clearly reach optima as $cos(0)$ equal to 1. 

If $v$ is tangential to $r$, we can just move $v$ from viewsphere at $p$ to our surface $s$.  

\subsection{Show $F_*(t)$ is orthogonal to $F_*(b)$}
Given that $b = p \times t$, we know for sure that $b$ is orthogonal to $t$. 

We have representation of $F_* (\pmb{t})$:
\[F_* (\pmb{t}) = \nabla r \pmb{p} + r\;\pmb{t}\]
Since $F_*$ is a tangent map from viewsphere to the surface, and $F_*(\pmb{t})$signifies the direction it is moving, when $t = \nabla \; r / \| \nabla \; r \|$, $F_*$ is moving tangentially to the surface. \\
Similarly, $F_*(b)$ is moving in the direction of $b$. 
Since $b$ is orthogonal to $t$, we can say that $F_*(t)$ is orthogonal to $F_*(b)$.

\subsection{Show N = T x B}
Given $\pmb{T}$, $\pmb{B}$, and $\pmb{t}$ as described, 
\[N = \pmb{T} \times \pmb{B}\]
We know that:
\[F_*(t) = \nabla r \; \pmb{p} + r \; \pmb{t}\]
\[F_*(b) = \nabla r \; \pmb{p} + r \; \pmb{b}\]
We also know that:
\[\pmb{b} = \pmb{p} \times \pmb{t} = \|\pmb{p}\| \; \|\pmb{t}\| \; sin(\pi/2)
 = \|\pmb{p}\| \; \|\pmb{t}\| \]
 Therefore, we have:
 \[F_*(t) \times  F_*(b) = (\nabla r \; \pmb{p} + r \; \pmb{t}) \times (\nabla r \; \pmb{p} + r  \|\pmb{p}\| \; \|\pmb{t}\|)\] 
 Since $\pmb{p}$ and $\pmb{t}$ are unite vectors, their norm is 1. 
 \[F_*(t) \times  F_*(b) = (\nabla r \; \pmb{p} + r \; \pmb{t}) \times (\nabla r \; \pmb{p} + r)\] 
 \[=\|\nabla r \; \pmb{p} + r\; \pmb{t}\| \;\|\nabla r \pmb{p}+r\|\]
 \[=\|\nabla r\| \; \pmb{t} + r \pmb{p}\]
Since $t$ and $p$ are unit vectors, the norm of them are going to be 1, hence:
\[\|F_*(t) \times  F_*(b)\| = \sqrt{\| \nabla \; r\|^2 + r^2}\]
\[\Rightarrow \pmb{N} = \frac{\|\nabla r\| \; \pmb{t} + r \pmb{p}}{\sqrt{\| \nabla \; r\|^2 + r^2}}\]

\subsection{examine the angle $\sigma$}
Given:
\[\pmb{N} = \frac{\|\nabla r\| \; \pmb{t} + r \pmb{p}}{\sqrt{\| \nabla \; r\|^2 + r^2}}\]
Since $\pmb{N}$ and $\pmb{p}$ are unit vectors, norm of them are 1. Therefore, we have:
\[sin\sigma = \|\pmb{N} \times \pmb{p}\|\]
\[= \frac{(\|\nabla r\| \; \pmb{t} + r \pmb{p}) \times \pmb{p}}{\sqrt{\| \nabla \; r\|^2 + r^2}}\]
We know that $\pmb{t} \times \pmb{p} = 1$ and $ \pmb{p} \times \pmb{p} = 0$, so 
this yields to:
\[\Rightarrow sin\sigma = \frac{(\|\nabla r\| }{\sqrt{\| \nabla \; r\|^2 + r^2}}\]
We know that dot product of two orthogonal vectors is 0, and dot product of two identical vector is the 2 norm.\\ 
\[ cos \sigma = \pmb{N} \cdot \pmb{p} = \frac{(\|\nabla r\| \; \pmb{t} + r \pmb{p}) \cdot \pmb{p}}{\sqrt{\| \nabla \; r\|^2 + r^2}}\]
\[\Rightarrow cos \sigma = \frac{r}{\sqrt{\| \nabla \; r\|^2 + r^2}}\]
With sin$\sigma$ and cos$\sigma$ solved, we can easily get tan $\sigma$:

\[tan \sigma = \frac{sin \sigma}{cos \sigma}\]
\[\Rightarrow tan \sigma = \frac{\| \nabla r\|}{r}\]

\subsection{Discuss slant and tilt}
It is much easier to show the distortion between the viewshere and our surface via slant and tilt. The angle, $\sigma$ acts as a map to convert $p$ and $t$ to $N$ and $T$. It is computationally much easier than using $\R^3$ coordinates.
The ellipse in Figure 1 can be represented as a collection of $N$ and $T$ that is coverted via $\sigma$.

\subsection{$F_*$}
\[F_* = \nabla r \; \pmb{p} + r\; \pmb{t}\]
\[= \nabla r \; (\pmb{t} \times \pmb{b}) + r\; \pmb{t}\]
\[N = (\pmb{b} \times \pmb{t})cos\sigma - \pmb{t} sin \sigma\]
\[T = (\pmb{b} \times \pmb{t}) sin\sigma +  \pmb{t} cos \sigma\]
Sorry, I don't quite understand the question, I understand the panelty.\\
\[ m = cos \sigma /r\]
\[M = 1/r\]

\subsection{t, b, m, and M}
The larger M and m is, the more distortion we will obtain from viewsphere. 

\end{document}
